\documentclass[10pt]{article}

\usepackage{amsmath, amssymb, amsthm, newtxtext, physics}
\title{\textbf{Conformality and Invariance}}
\date{}
\usepackage[margins = 0.25in]{geometry}
\theoremstyle{plain} 
\newtheorem{definition}{Definition}
\newtheorem{theorem}{Theorem}
\newtheorem{example}{Example}
\newtheorem{observation}{Observation}
\begin{document}
	\maketitle 
	
\noindent \textbf{\textit{Goal:}} Define a notion of distance that is preserved under holomorphic \& conformal maps. Let's see why the regular notion of distance doesn't give us what we want.



\subsection*{The Jacobian of a Holomorphic Function}


Let $U \subseteq \mathbb{C}$ be an open set, $P \in U$ a fixed point, and $f: U \to \mathbb{C}$ a holomorphic function on $U$. For $f(x + iy) = u(x, y) + iv(x,y)$, we can consider $f$ as a mapping $(x, y) \to (u, v)$, where we get the real Jacobian matrix of $f$ at $P$: 
	\begin{eqnarray}
		J(P) = \begin{pmatrix}u_x(P) & u_y(P) \\ v_x(P) & v_y(P)\end{pmatrix}
	\end{eqnarray}
Since $f$ is holomorphic, by the Cauchy-Riemann equations we know that
	\begin{equation}
		\boxed{u_x = v_y, \qquad u_y = -v_x}
	\end{equation}
Hence, we can simplify (1) in the following way:
	\begin{align*}
		J(P) &=  \begin{pmatrix}u_x(P) & u_y(P) \\ v_x(P) & v_y(P)\end{pmatrix} \\
		&= \begin{pmatrix}u_x(P) & u_y(P) \\ -u_y(P) & u_x(P)\end{pmatrix} \\
		&= \underbrace{\sqrt{u_x(P)^2 + u_y(P)^2}}_{=: h(P)} \cdot \underbrace{ \begin{pmatrix} \frac{u_x(P)}{\sqrt{u_x(P)^2 + u_y(P)^2}} & \frac{u_y(P)}{\sqrt{u_x(P)^2 + u_y(P)^2}} \\ \frac{-u_y(P)}{\sqrt{u_x(P)^2 + u_y(P)^2}} & \frac{u_x(P)}{\sqrt{u_x(P)^2 + u_y(P)^2}} \end{pmatrix}} _{=: \mathcal{J}(P)}
	\end{align*}
Then, 
	\begin{equation}
		J(P) \equiv h(P) \cdot \mathcal{J(P)}.
	\end{equation}

We now make the following observations:
	\begin{observation} ~
		\begin{itemize}
			\item[(1)] $\mathcal{J}(P)$ is an orthogonal matrix.
			\item[(2)] The rows of $\mathcal{J}(P)$ form an orthonormal basis for $\mathbb{R}^2$ with positive orientation. 
			\item[(3)] For $\mathbf{x}, \mathbf{y} \in \mathbb{R}^2$, $$\norm{J(P) \mathbf{x} - J(P) \mathbf{y}} = h(P) \norm{\mathbf{x} - \mathbf{y}}.$$
			
			\item[(4)] If $\angle(\mathbf{x}, \mathbf{y})$ denotes the angle between two vectors $\mathbf{x}, \mathbf{y} \in \mathbb{R}^2$, then $$\angle (\mathbf{x}, \mathbf{y}) = \angle(J(P) \mathbf{x}, J(P) \mathbf{y}).$$
		\end{itemize}
	\end{observation}
	\begin{proof} ~
		\begin{itemize}
			\item[(1)] We have,
				\begin{align*}
					\left[\mathcal{J}(P) \cdot \mathcal{J}(P)^T\right]_{ij} &= [\mathcal{J}(P)]_{i1} [\mathcal{J}(P)^T]_{1j} + [\mathcal{J}(P)]_{i2} [\mathcal{J}(P)^T]_{2j} \\
					&= [\mathcal{J}(P)]_{i1} [\mathcal{J}(P)]_{j1} + [\mathcal{J}(P)]_{i2} [\mathcal{J}(P)]_{j2}
				\end{align*}
			So, by direct computation,
				\begin{align*}
					\left[\mathcal{J}(P) \cdot \mathcal{J}(P)^T\right]_{11} &= [\mathcal{J}(P)]_{11}^2 + [\mathcal{J}(P)]_{12}^2 \\
					&= \left(\frac{u_x(P)}{\sqrt{u_x(P)^2 + u_y(P)^2}}\right)^2 + \left(\frac{u_y(P)}{\sqrt{u_x(P)^2 + u_y(P)^2}}\right)^2 \\ \\
					&= \frac{u_x(P)^2}{u_x(P)^2 + u_y(P)^2} + \frac{u_y(P)^2}{u_x(P)^2 + u_y(P)^2} \\
					&= \frac{u_x(P)^2 + u_y(P)^2}{u_x(P)^2 + u_y(P)^2} \\
					&= 1 \\ 
					\\
					\left[\mathcal{J}(P) \cdot \mathcal{J}(P)^T\right]_{12} &= [\mathcal{J}(P)]_{11} \underbrace{[\mathcal{J}(P)]_{21}}_{= - [\mathcal{J}(P)]_{12}} + [\mathcal{J}(P)]_{12} \underbrace{[\mathcal{J}(P)]_{22}}_{= [\mathcal{J}(P)]_{11}} \\
					&= -[\mathcal{J(P)}]_{11} [\mathcal{J}(P)]_{12} + [\mathcal{J(P)}]_{11} [\mathcal{J}(P)]_{12} \\
					&= 0 \\
					\\
					\left[\mathcal{J}(P) \cdot \mathcal{J}(P)^T\right]_{21} &= [\mathcal{J}(P)]_{21} [\mathcal{J}(P)]_{11} + [\mathcal{J}(P)]_{22}[\mathcal{J}(P)]_{12} \\
					&= 0 \\
					\\
						\left[\mathcal{J}(P) \cdot \mathcal{J}(P)^T\right]_{22} &= [\mathcal{J}(P)]_{21}^2 + [\mathcal{J}(P)]_{22}^2 \\
						&= (-[\mathcal{J}(P)]_{12})^2 + [\mathcal{J}(P)]_{22}^2 \\
						&= 1.
				\end{align*}
		Hence, $$\mathcal{J}(P) \cdot \mathcal{J}(P)^T = \begin{pmatrix} 1 & 0 \\ 0 & 1\end{pmatrix}.$$
			\item[(2)] Just view the above computation, as we have shown that $$[\mathcal{J}(P)]_{11} [\mathcal{J}(P)]_{21} + [\mathcal{J}(P)]_{12} [\mathcal{J}(P)]_{22} = 0.$$ To show that the rows form an orthonormal basis with \textit{positive} orientation, we have
				\begin{align*}
					([\mathcal{J}(P)]_{11} \mathbf{i} + [\mathcal{J}(P)]_{12} \mathbf{j}) &\times ([\mathcal{J}(P)]_{21} \mathbf{i} + [\mathcal{J}(P)]_{22} \mathbf{j}) = \begin{vmatrix} [\mathcal{J}(P)]_{11} &  [\mathcal{J}(P)]_{12} \\  [\mathcal{J}(P)]_{21} & [\mathcal{J}(P)]_{22}\end{vmatrix} \mathbf{k} \\
					&= ([\mathcal{J}(P)]_{11} [\mathcal{J}(P)]_{22} - [\mathcal{J}(P)]_{12} [\mathcal{J}(P)]_{21})\mathbf{k} \\
					&= ([\mathcal{J}(P)]_{11}^2 + [\mathcal{J}(P)]_{12}^2) \mathbf{k}\\
					&= +\mathbf{k}
				\end{align*}
			
			\item[(3) - (4)] Since $\mathcal{J}(P)$ is an orthogonal matrix, it preserves lengths of vectors and angles between them, and thus (3) follows. A scaling transformation, which is what $h(P)$ is, obviously preserves angles, and so (4) follows. 
		\end{itemize}
	\end{proof}

\subsection*{Conformal Mappings of the Unit Disk}
\noindent Since the author of this book says that\textbf{\textit{ conformal mappings}} are characterized by the fact that they infinitesimally 
	\begin{itemize}
		\item[(i)] preserve angles, and
		\item[(ii)] preserve length (up to a scalar factor) 
	\end{itemize}
we have just shown that
	\begin{theorem}
		The Jacobian $J(P)$ of a holomorphic map $f: U \subseteq \mathbb{C} \to \mathbb{C}$ at a point $P \in U$ is a conformal map. 
	\end{theorem}
\noindent 
	Despite this, $J(P)$ fails to preserve distances. So we have an example of a conformal map that doesn't preserve Euclidean distance. Let's reduce our view to conformal mappings of the unit disk for the time being. 

	When $U$ is the unit disk in $\mathbb{C}$, we have a nice classification theorem for all conformal mappings on $U$. 
	
	\begin{theorem}[The Conformal Mappings of the Unit Disk]
		Let $D = D(0, 1)$ denote the unit disk in $\mathbb{C}$. Then a conformal mapping on $D$ is either
			\begin{itemize}
				\item[(i)] A rotation $\rho_\lambda : z \mapsto e^{i \lambda} \cdot z$, $0 \leq \lambda < 2 \pi$;
				
				\item[(ii)] A Möbius transformation of the form $\varphi_a : z \mapsto [z - a]/[1 - \overline{a} z]$, $a \in \mathbb{C}$, $|a| < 1$; or 
				
				\item[(iii)] A composition of mappings of type (i) and (ii).
			\end{itemize}
	\end{theorem}

\subsection*{Constructing the Poincaré Metric}
To "discover" the Poincaré metric, we start with what we want (invariance of the metric under conformal mappings) and go from there. 
	\begin{enumerate}
		\item Given any vector, sourcing from a point $P$ in the direction of $\mathbf{v}$, denote its \textit{length} by $|\mathbf{v}|_P$. 
		
		\item Declare that the length of $\mathbf{e} = (1, 0)$ is 1. So, $|\mathbf{e}|_0 = 1$. 
		
		\item If $\phi$ is a conformal self-map of the disk, then we claim that the condition
			\begin{equation}
				|\mathbf{v}|_P = |\phi_\ast(P) \mathbf{v}|_{\phi(P)},
			\end{equation}
		where $\phi_\ast(P)\mathbf{v} := \phi'(P) \cdot \mathbf{v}$ denotes the \textit{\textbf{push-forward}} by $\phi$ of the vector $\mathbf{v}$, is equivalent to invariance under this new metric we're defining. Why? Since $\phi'(P)$ is the Jacobian of $\phi$ at $P$, then by our previous work, this action amounts to a rotation and a scaling. We're saying the above condition encapsulates that, after that rotation and scaling, the length of our vector stays the same. Now, we are \textit{requiring} that $|\cdot|$ is such that (4) works for any conformal self-map of the disk $\phi$. We're using this condition to derive an explicit formula for $|\cdot|$.
		
		
		\item Let $\phi(z) = e^{i \lambda} \cdot z$. Then, first of all $$\phi(z) = (\cos \lambda + i \sin \lambda)(x + iy) = x \cos \lambda  - y \sin \lambda + i (x \sin \lambda + y \cos \lambda)$$ so then, 
			$$\phi'(z) = \begin{bmatrix} \cos \lambda & -\sin \lambda \\ \sin \lambda & \cos \lambda \end{bmatrix}$$  
		Condition (4) says that we must have $$1 = |\mathbf{e}|_0 = |\phi_\ast(0) \mathbf{e}|_{\phi(0)} = |e^{i \lambda} \cdot \mathbf{e}|_0 = |e^{i \lambda}|_0$$ Therefore, we conclude the length of any Euclidean unit vector based at the origin is 1 in this new invariant metric.
		
		\item Let $\phi(z)$ be the Möbius transformation $$\psi(z) = \frac{z + a}{1 + \overline{a} z}$$ with $a \in \mathbb{C}$, $|a| < 1$. Let $\mathbf{v} = e^{i \lambda}$ be a unit vector at the origin. Then, with $$\psi'(0) = 1 - |a|^2$$ and (4), we have $$1 = |\mathbf{v}|_0 = |\psi_\ast(0) \mathbf{v}|_{\psi(0)} = |(1 - |a|^2) \cdot \mathbf{v}|_a$$ Hence, $$|\mathbf{v}|_a = \frac{1}{1 - |a|^2}.$$
		We are ready.
	\end{enumerate}

	\begin{definition}[The Poincaré Metric]
		Let $P$ be a point of the unit disk $D$ and let $\mathbf{v}$ be any vector based at that point. Then, $$\boxed{|\mathbf{v}|_P = \frac{\norm{\mathbf{v}}}{1 - |P|^2}}$$ is called the \textbf{\textit{Poincaré metric}}. 
	\end{definition}

	\noindent Note that, with the above definition, $|\mathbf{v}|_P$ as $P$ approaches $\partial D$. 
\end{document}